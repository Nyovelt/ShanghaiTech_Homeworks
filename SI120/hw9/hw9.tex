% !TEX program = xelatex

\documentclass{article}
\usepackage{amsfonts,amssymb}
\usepackage{amsmath}
\usepackage{amsthm}
\usepackage[left=1.0cm,right=1.0cm,top=1.3cm,bottom=1.3cm]{geometry}
\usepackage{enumerate}
\usepackage{fancyhdr}
\usepackage{ctex}
\usepackage{xpatch}
\usepackage{graphicx} %插入图片的宏包
\usepackage{float} %设置图片浮动位置的宏包
\usepackage{subfigure} %插入多图时用子图显示的宏包
\usepackage{listings}
\usepackage{color}
\usepackage{amssymb,mathrsfs,amsmath}
\usepackage{bbding}
\usepackage{mathrsfs}

\definecolor{dkgreen}{rgb}{0,0.6,0}
\definecolor{gray}{rgb}{0.5,0.5,0.5}
\definecolor{mauve}{rgb}{0.58,0,0.82}

\lstset{frame=tb,
  language=Python,
  aboveskip=3mm,
  belowskip=3mm,
  showstringspaces=false,
  columns=flexible,
  basicstyle={\small\ttfamily},
  numbers=none,
  numberstyle=\tiny\color{gray},
  keywordstyle=\color{blue},
  commentstyle=\color{dkgreen},
  stringstyle=\color{mauve},
  breaklines=true,
  breakatwhitespace=true,
  tabsize=3
}


\newtheoremstyle{break}
    {\topsep}{\topsep}%
    {\itshape}{}%
    {\bfseries}{}%
    {\newline}{}%
\theoremstyle{break}
\newtheorem*{solution*}{\textbf{Solution:} }
%\newtheorem*{proof*}{\textbf{Proof:}}
\makeatletter

\AtBeginDocument{\xpatchcmd{\@thm}{\thm@headpunct{.}}{\thm@headpunct{}}{}{}}
\makeatother

\pagestyle{fancy} 
\lhead{Name}
\chead{\textbf{Discrete Mathematics:  Homework 9}}
\rhead{2020.5.7}
\renewcommand{\baselinestretch}{1.5}

\title{Discrete Mathematics:  Homework 9}
\author{Name  \quad  \quad ID: Number}
\date{2020.5.7}


\begin{document}
\maketitle
\begin{enumerate}
        \item Suppose that a language has 38 letters in its alphabet $\mathcal{A}$.  Suppose that $A, B \in \mathcal{A}$.  Thelength of a word is the number of letters it has.
        \begin{enumerate}
                \item How may words have length 7, having A as the third letter ? 
                \item How many words of length 4 such that the letter B appears exacly twice?
        \end{enumerate}
        \begin{solution*}
                \begin{enumerate}
                        \item 
                        For the rest of 6 numbers, each digit has 38 possibilities. And for the third digit, it has only one possibility 'A' .\\
                        So the answer is,
                        $  38^6 $
                        \item
                        For 4 digits, there has 2 digits which can be 'B', so the possibility is $C_4^2$, for the rest 2, each digit has 37 possibilities. (Because they can't be 'B') \\
                        So the answer is $C_4^2 \cdot 37^2$
                \end{enumerate}
        \end{solution*}
        \vspace{10mm}
        \item A  manager  selects  a  football  team  from  a  squad  of  20  players.   The  squad  has  17 outfield players and 3 goalkeepers.  The team should have 1 goalkeeper and 10 outfieldplayers.
        \begin{enumerate}
                \item How many selections are possible?  (the selection does NOT include shirt number,the position of outfield players, etc...)
        \end{enumerate}
        \begin{solution*}
                 $$C_{17}^{10} \cdot C_{3}^{1}$$
        \end{solution*}
        \newpage
        \item Suppose  we  have  a  bowl  with  red  marbles,  green  marbles,  yellow  marbles,  purplemarbles and blue marbles in it (the number of marbles of a given colour can be 0).Marbles of the same colour are indistinguishable.
        \begin{enumerate}
                \item If the bowl has ten marbles, how many possibilities are there?
                \item If the bowl has 13 marbles and the bowl does not contain marbles of all 5 colours,how many possibilitles are there?
        \end{enumerate}
        \begin{proof}
                \begin{enumerate}
                        \item For ten marbles, we can consider:
                        | o o o o o | o | o o | o o |, \\
                        each '|' has 14 slot, so the answer is
                        $$ C_{14}^4 = 1001 $$
                        \item The answer is all possibilities without 5 
                        $$  C_{17}^4 - C_{12}^4 = 1885    $$
                \end{enumerate}
        \end{proof}
        \vspace{10mm}
        \item Prove  that  for  any  positive  integer $n$,  there  exists  infinitely  many  positive  integers $k$, such that $kn$ has only 0 and 7 in its decimal expansion (for example:  70700077).Explain your answer with as much detail as possible
        \begin{proof}
                For integers 7, 77, 777,$\underbrace{777 \dots 777}_{n+1}$, there at least have 2 intergers such that $n \bmod \underbrace{77 \dots 77}_{a} $ and $n \bmod \underbrace{77 \dots 77}_{b}$ by pigeonhole principle.\\
                The same holds for $\underbrace{777 \dots 777}_{n+2} \cdots \underbrace{777 \dots 777}_{2n+2}$ and $\underbrace{777 \dots 777}_{2n+3} \cdots \underbrace{777 \dots 777}_{3n+3}$ and $\dots$\\
                So we can find infinite $a,b \in \{ 77 \cdots 77 \}$ such that $n \bmod a$  and $n \bmod b$ \\
                we also have $n \bmod |a-b|$ and $|a-b| \in \{ 7 \dots 0 \dots 7\}$\\
                So there are infinite $k$ 
        \end{proof}
\end{enumerate}
\end{document}