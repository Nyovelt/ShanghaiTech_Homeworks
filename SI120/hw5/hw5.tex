% !TEX program = xelatex

\documentclass{article}
\usepackage{amsfonts,amssymb}
\usepackage{amsmath}
\usepackage{amsthm}
\usepackage[left=1.0cm,right=1.0cm,top=1.3cm,bottom=1.3cm]{geometry}
\usepackage{enumerate}
\usepackage{fancyhdr}
\usepackage{ctex}
\usepackage{xpatch}
\usepackage{graphicx} %插入图片的宏包
\usepackage{float} %设置图片浮动位置的宏包
\usepackage{subfigure} %插入多图时用子图显示的宏包


\newtheoremstyle{break}
    {\topsep}{\topsep}%
    {\itshape}{}%
    {\bfseries}{}%
    {\newline}{}%
\theoremstyle{break}
\newtheorem*{solution*}{\textbf{Solution:} }
%\newtheorem*{proof*}{\textbf{Proof:}}
\makeatletter

\AtBeginDocument{\xpatchcmd{\@thm}{\thm@headpunct{.}}{\thm@headpunct{}}{}{}}
\makeatother

\pagestyle{fancy}
\lhead{Name}
\chead{\textbf{Discrete Mathematics:  Homework 5}}
\rhead{2020.4.8}
\renewcommand{\baselinestretch}{1.5}

\title{Discrete Mathematics:  Homework 5}
\author{Name  \quad  \quad ID: Number}
\date{2020.4.8}


\begin{document}
\maketitle
\begin{enumerate}
\item Let $a \in \mathbb{Z}, b \in \mathbb{{Z}^+}$ and $x \in \mathbb{{R}}$. Show  that  there  exist  unique $q,r \in \mathbb{{Z}}$ such that $a = bq+r$ and $x \leq r < x+b$.
\begin{proof}
        For $ a - \lceil x \rceil = bq +r^{'}$, $q,r^{'}$ is unique and exists. $r^{' }\in [0,b)$. Because r is an integer, $r^{'} \in [0,b-1]$\\
        $a = bq + r ^{'}+ \lceil x \rceil $, $q,r^{'}$ is unique and exists.$r ^{'}\in [0,b-1]$. \\
        Let $r = r^{'} + \lceil x \rceil  $, $\lceil x \rceil  \in [x, x+1 )$\\
        Because $r^{'}$ is unique and exists, $\lceil x \rceil $ is known, $r$ is unique and exists.\\
        $r \in [\lceil x \rceil , b+\lceil x \rceil -1 )$ , we have $ r \in [ x, x+b )$
\end{proof}
\vspace{30mm}
\item Let $a,b > 1$ be relatively prime integers.  Show that if $a|n$ and $b|n$, then $ab|n$.
\begin{proof}
Proof by contradiction:\\
let $S = \{  x | x \in \mathbb{Z}  , a|x \ and \ b|x \}$  and $n$ is the smallest elements of $S$ and $n$ exists.\\
Suppose $n = k(ab) + r , r \in (0, ab)$ , $a|n$ and $b|n$\\
By $a|n$, we have $\frac{n}{a} = kb + \frac{r}{a} , r = aq ,q \in \mathbb{Z}^+ $ \\
By $b|n$, we have $\frac{n}{b} = ka + \frac{r}{b} , r = bp, p \in \mathbb{Z}^+ $ \\
So, $a|r$ and $b|r$. \\
By surmise, $r \in S$ and $r < ab < n $\\
So, $n$ doesn't exists as the smallest elements in $S$.\\
So, $ab | n$.\\
\end{proof} 
\newpage
\item Let $a, b_1, b_2, \dots , b_k \in \mathbb{Z}^+$. Show that $gcd(a, b_1b_2 \dots b_k) = 1 $ iff $gcd(a, b_i) =1$ for every $i \in {k}$
\begin{proof}
    By undamental theorem of arithmetic , we have \\
    $ a = p_{0_1}^{e_{0_1}} p_{0_2}^{e_{0_2}} \dots p_{0_r}^{e_{0_r}}$\\
    $ b_1 = p_{1_1}^{e_{1_1}} p_{1_2}^{e_{1_2}} \dots p_{1_r}^{e_{1_r}}$\\
    \dots\\
    $ b_k = p_{k_1}^{e_{k_1}} p_{k_2}^{e_{k_2}} \dots p_{k_r}^{e_{k_r}}$\\
    where $p_{k_i}$ are primes and $e_{k_i} \geq 1$\\
    Prove \textbf{if}:\\
    if $gcd(a, b_i) =1$ for every $i \in {k}$ \\
    then $\forall m,n \in \mathbb{Z}^+ $,$ p_{i_m} \neq p_{0_n}$ \\
    and $ b_1b_2 \dots b_k  =  p_{1_1}^{e_{1_1}} p_{1_2}^{e_{1_2}} \dots p_{1_r}^{e_{1_r}} \dots p_{k_1}^{e_{k_1}} p_{k_2}^{e_{k_2}} \dots p_{k_r}^{e_{k_r}} $\\
    doesn't have same divisor ${p}$ with $a$ except $1$.\\
    So $gcd(a, b_1b_2 \dots b_k) = 1 $ .\\
    Prove \textbf{only if}:
    if $gcd(a, b_1b_2 \dots b_k) = 1 $ \\
    $\forall m,n \in \mathbb{Z}^+, i \in \{k\} $ ,  $p_{0_n} \neq p_{i_m}$\\
    so $b_i$ doesn't have same divisor ${p_{i_m}}$ with $a$ except $1$.\\
    $gcd(a, b_i) =1$ for every $i \in {k}$\\
\end{proof}
\vspace{10mm}
\item Let $x \in \mathbb{R}$ and $n \in \mathbb{Z}^+$. Show that $\lfloor \frac{ \lfloor x \rfloor }{n} \rfloor = \lfloor \frac{x}{n} \rfloor$
\begin{proof}
    $ \forall x \in \mathbb{R}, \exists a \in \mathbb{Z}, \exists \epsilon \in [0,1) $ , such that $x = a + \epsilon ,\lfloor x \rfloor = a$ \\
    By division algorithm, there exists unique $p \in \mathbb{Z}, r \in (0, n)$ , such that $a=pn + r$ \\
    Because $a \in\mathbb{Z}$, so $r \in \mathbb{Z}$ , we have $r \in (0, n-1] $\\
    For the left side of the equation, $\lfloor \frac{ \lfloor x \rfloor }{n} \rfloor = \lfloor \frac{a}{n} \rfloor = p $\\
    For the right side of the equation, $ \lfloor \frac{x}{n} \rfloor = \lfloor \frac{a + \epsilon}{n} \rfloor = \lfloor \frac{a}{n}  + \frac{\epsilon}{n}\rfloor = \lfloor p + \frac rn + \frac{\epsilon}{n} =  \lfloor p +\frac{r + \epsilon}{n} \rfloor $\\
    Because $ r \leq n-1, \epsilon < 1$, $r + \epsilon < n $ , so $\lfloor p +\frac{r + \epsilon}{n} \rfloor  = p $ \\
    Left side = Right side\\
\end{proof}
\newpage
\item Let $a,b \in \mathbb{Z}, n \in \mathbb{Z}^+$ and $a \equiv b \ (mod \ n) $. Let $c_0,c_1, \dots , c_k \in \mathbb{Z}$, where $k \in \mathbb{Z}^+$. Show that $c_0 + c_1a + \dots + c_k a^k \equiv  c_0 + c_1b + \dots + c_k b^k \  (mod \ n)$.
\begin{proof}
By division algorithm, $a = q_a n + r_a$ and $b = q_b  n+ r _b$  $r_a,r_b \in \mathbb{Z}, r_a \in [ 0, n ) ,r_b \in [ 0, n )$\\
Because $a \equiv b \ (mod \ n) $, we have $r_a = r_b$\\
$a^i =   ( q_a n + r_a )^i =  f(q_a, r_a)n + r_a^i  \equiv r_a^i \ (mod \ n)$\\
$b^i =   ( q_b n + r_b )^i =  f(q_b, r_b)n + r_b^i  \equiv r_b^i \ (mod \ n)$\\
where $f(x,y) = $ 
\[
    \sum_{j=0}^{j < i} C_i^j x^{i-j} y^j
\]
Because $ r_a^i \equiv r_b^i ( mod \ n) $ , so we have  $ a_i \equiv b_i\  (mod  \ n )$\\
which equals to,
\[c_0 + c_1a + \dots + c_k a^k \equiv  c_0 + c_1b + \dots + c_k b^k \  (mod \  n)\]
\end{proof}
\vspace{10mm}
\item Let $p$ be a prime and $p \notin \{2,5\} $. Show that  $p$ divides infinitely many elements of the set $  \{       9,99,999,9999,99999, \dots    \}$.
\begin{proof}
$ [10]_p = 10 + np$ and $p \notin \{2,5\} $, we have $gcd([10]_p,p)=1$\\
By Fermat's little theorem, we have ${ [10]_p} ^ {p-1}\equiv 1 \ (mod \ p) $\\
which is $p |({ [10]_p} ^ {p-1} -1 )\Rightarrow p | ((10 + np) ^{p-1} -1) $.\\
$\Rightarrow  p | (10^{p-1}+  \sum_{i=0}^{i<p-1}  C_{p-1}^{i}10^{i} (np)^{p-1-i} - 1)  $ \\
$ \Rightarrow p | (10^{p-1} -1)  $
\end{proof}
\end{enumerate}
\end{document}