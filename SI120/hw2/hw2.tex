% !TEX program = xelatex

\documentclass{article}
\usepackage{amsmath}
\usepackage{amsthm}
\usepackage[left=1.0cm,right=1.0cm,top=1.3cm,bottom=1.3cm]{geometry}
\usepackage{enumerate}
\usepackage{fancyhdr}
%\usepackage{ctex}
\usepackage{xpatch}
\usepackage{graphicx} %插入图片的宏包
\usepackage{float} %设置图片浮动位置的宏包
\usepackage{subfigure} %插入多图时用子图显示的宏包
\usepackage{amsfonts}


\newtheoremstyle{break}
    {\topsep}{\topsep}%
    {\itshape}{}%
    {\bfseries}{}%
    {\newline}{}%
\theoremstyle{break}
\newtheorem*{solution*}{\textbf{Solution:} }
%\newtheorem*{proof*}{\textbf{Proof:}}
\makeatletter

\AtBeginDocument{\xpatchcmd{\@thm}{\thm@headpunct{.}}{\thm@headpunct{}}{}{}}
\makeatother

\pagestyle{fancy}
\lhead{Your Name}
\chead{\textbf{Discrete Mathematics:  Homework 2}}
\rhead{2020.3.19}
\renewcommand{\baselinestretch}{1.5}

\title{Discrete Mathematics:  Homework 2}
\author{Your Name \quad  \quad ID: Your ID}
\date{2020.3.19}




\begin{document}
    \maketitle
    \begin{enumerate}
        \item 
        Let p and q be  two  propositional  variables.   Up  to  logical  equivalence  (i.e.,  if A $\equiv$ B, then we consider A and B has the same formula), how many different formulas in p, q are there?  Write down all of them.
        \begin{solution*}
            There are 15 different formulas. \newline
            \begin{center}
                \begin{tabular}{|c|c|c|}
                    \hline
                    $\lnot ( \lnot p ) \equiv p $ & $ p \land p \equiv p$  & $p \lor p \equiv p $\\
                    \hline
                    $p \land q \equiv q \land p$ & $ p \lor q \equiv q \lor p $ & $ \lnot(p \land q)\equiv (\lnot p)\lor (\lnot q)$\\
                    \hline
                    $ \lnot(p \lor q)\equiv (\lnot p)\land (\lnot q)$ & $p \lor ( p \land q) \equiv p$ & $p \lor ( p \land q) \equiv p$\\
                    \hline
                    $p \to q \equiv \lnot p \lor q$ & $p \to q \equiv \lnot q \land \lnot q$ & $p\leftrightarrow q \equiv (\lnot p \lor q) \land \lnot(p \lor \lnot q ) $\\
                    \hline
                    $p\leftrightarrow q \equiv ( p \land q) \lor (\lnot p \land \lnot q ) $ & $p\leftrightarrow q \equiv (p \to q ) \land ( q \to p) $ & $ p \leftrightarrow q \equiv \lnot q \leftrightarrow \lnot p$\\
                    \hline
                \end{tabular}
            \end{center}
        \end{solution*}
        
        \newpage
        \item 
        Let ↑ be a binary operation defined by the following truth table:
        \newline
        \begin{center}
            \begin{tabular}{|c|c|c|}
                \hline
                p &q &p ↑ q\\
                \hline
                T &T& F\\
                \hline
                T& F &T\\
                \hline
                F& T& T\\
                \hline
                F &F &T\\
                \hline
            \end{tabular}
        \end{center}
        Represent every formula in Question 1 with an expression where the only operation is ↑.  For
        example,  for  the  formula  ¬p $\lor$ ¬q,  we  have  the  representation  ¬p $\lor$ ¬q  $\equiv$ p  ↑ q,  where  
        the only operation in the right-hand expression p ↑ q  is ↑ and none of the other operations such
        as $\lnot$, $\land$, $\lor$, $\to$ or $\leftrightarrow$ appears.
        \begin{solution*}\leavevmode\\
        \begin{center}
            \begin{tabular}{|c|c|c|}
                \hline
                1 & $\lnot ( \lnot p ) \equiv p $ &$\lnot ( \lnot p) \equiv  p$ \\
                \hline
                2 &  $p \land p \equiv p$ &$p \land p \equiv p $ \\
                \hline
                3 &  $p \lor p \equiv p $ & $p \lor p \equiv p$ \\
                \hline
                4 & $p \land q \equiv q \land p$ & $p \land q \equiv (p \uparrow q ) \uparrow (p \uparrow q)$\\
                \hline
                5 & $ p \lor q \equiv q \lor p $ & $p \lor q \equiv  (p \uparrow   p) \uparrow (q \uparrow q)$  \\
                \hline
                6 & $ \lnot(p \land q)\equiv (\lnot p)\lor (\lnot q)$ & $\lnot(p \land q)\equiv  p \uparrow q$ \\
                \hline
                7 & $ \lnot(p \lor q)\equiv (\lnot p)\land (\lnot q)$ & $\lnot(p \lor q)\equiv (p \uparrow p) \uparrow (q \uparrow q) \uparrow (p \uparrow p) \uparrow (q \uparrow q)  $\\
                \hline
                8 & $p \lor ( p \land q) \equiv p$ & $p \lor ( p \land q) \equiv p$  \\
                \hline
                9 & $p \lor ( p \land q) \equiv p$ & $p \lor ( p \land q) \equiv p$ \\
                \hline
                10 & $p \to q \equiv \lnot p \lor q$ & $p \to q \equiv p \uparrow (q \uparrow q) $ \\
                \hline
                11 & $p\leftrightarrow q \equiv (\lnot p \lor q) \land \lnot(p \lor \lnot q ) $ & $p\leftrightarrow q \equiv  (p \uparrow q) \uparrow (p \uparrow p) \uparrow (q \uparrow q)$\\
                \hline
                \end{tabular}
        \end{center}
        \end{solution*}
        \newpage

        \item 
        Show that (P $\land$ Q $\land$ S) $\lor$ (P $\land$¬Q $\land$ ¬R) $\lor$ (P $\land$ Q $\land$ $\lnot$ S) $\lor$ $\lnot$  (P $\land$ R $\to$ Q) $\equiv$ P using
    the  logical  equivalences on  page  8-10 of  lec3.pptx.  (Hint:  see  page  11,  12 for  an 
    example)
    \begin{solution*}
        \begin{equation}
            \begin{aligned}
                &\quad (P \land Q \land S) \lor (P \land \lnot Q \land \lnot R) \lor (P \land Q \land \lnot S) \lor \lnot(P \land R \to Q) \\
                &\equiv (P \land Q \land S) \lor ( P \land Q \land \lnot S) \lor (P \land \lnot Q \land \lnot R) \lor ( P \land R \land \lnot Q ) \\
                & \equiv(P \land Q) \lor ( P \land \lnot Q \land \lnot R) \lor ( P \land R \land \lnot Q ) \\
                & \equiv(P \land \lnot Q \land \lnot R) \lor (P \land \lnot Q \land R) \lor (P \land Q) \\
                &\equiv (P \land Q) \lor (P \land \lnot Q)\\
                & \equiv P
            \end{aligned} 
        \end{equation}
    \end{solution*}

\vspace{50mm}

    \item 
    Show that $(P \lor Q \to R \land S ) \land (S \lor W \to U) \land P \Rightarrow U \lor V$ 
    using the tautological implications  (and  the  resulting  valid  argument  forms)  on  page  3  of  lec4.pptx.   (Hint:  
    see page 12, 13 for an example)
    
    \begin{solution*}
        \begin{equation}
            \begin{aligned}
                & \quad (P \lor Q \to R \land S ) \land (S \lor W \to U) \land P \\
                & \equiv (\lnot (P \lor Q) \lor (R \land S) \land (\lnot S \land \lnot W \lor U)) \\
                & \equiv (\lnot P \land \lnot Q \land P) \lor  (R \land S  \land P) \land (\lnot S \land \lnot W \lor U) \\
                & \equiv (R \land S \land P \land \lnot S \land \lnot W ) \lor (R \land S \land P \land U) \\
                & \equiv R \land S \land P \land U \\
                & \Rightarrow U \\
                & \Rightarrow U \lor V\\
            \end{aligned}
        \end{equation}
    \end{solution*}
    \newpage
    \item 
        Use the tautological implications (and the resulting valid argument forms) on
    page 3 of lec4.pptx to show that the premises If it does not rain or if it is not foggy, then
    the sailing race will be held and the lifesaving demonstration will go on," If the sailing race
    is held, then the trophy will be awarded," and " The trophy was not awarded" imply "the
    conclusion It rained."
    \begin{solution*}\leavevmode
        \begin{enumerate}[]
            \item p: It rained. 
            \item q: It's foggy.
            \item r: The sailing race will be held.
            \item s: The lifesaving demonstration will go on.
            \item t: The trophy will be awarded.
        \end{enumerate}
        so, we have

            \begin{equation}
                \begin{aligned}
                    \lnot p \lor \lnot q &\to r \land s \\
                    r &\to t \\
                    \lnot t &\Rightarrow p\\
                \end{aligned}
        \end{equation}
        so,
        \begin{equation*}
            \begin{aligned}
                & \quad (\lnot p \lor \lnot q \to r \land s) \land (r \to t) \land (\lnot t) \\
                & \equiv( p \land q \lor r \land s )\land (\lnot r \lor t) \land \lnot t \\
                & \equiv( p \land q \lor r \land s) \land ( \lnot r \land \lnot t) \\
                & \Rightarrow p
            \end{aligned}
        \end{equation*}
    \end{solution*}


    \newpage
    \item   Suppose that the following two premises are true:
    \begin{enumerate}
        \item  Math is hard or Leibniz doesn't like Math";
        \item  If SI120 is easy, then Math is not hard".
    \end{enumerate}
    Which of the following conclusions are true under the above premises?
    \begin{enumerate}
        \item  If Leibniz likes Math, then SI120 is not easy."
        \item  If SI120 is not easy, then Leibniz doesn't like Math."
        \item  Math is not hard or SI120 is not easy."
        \item Leibniz doesn't like Math, then either SI120 is not easy or Math is not hard."
    \end{enumerate}
    \textbf{Justify your answers.}
    \begin{solution*}\leavevmode\\
        \begin{enumerate}[]
            \item p: Math is hard.
            \item q: Leibniz like Math.
            \item r: SI120 is easy.
        \end{enumerate}
        so, we have
        \begin{equation*}
            \begin{aligned}
                p &\lor \lnot q \\
                r &\to \lnot p \\
            \end{aligned}
        \end{equation*}
        \begin{enumerate}
            \item  
            \begin{equation*}
                \begin{aligned}
                    & \quad (p \lor \lnot q) \land (r \to \lnot p) \Rightarrow (q \to \lnot r) \\
                    & let \quad A = ( p \lor \lnot q) \land (r \to \lnot p) ,  B = (q \to \lnot r) \\
                    & A^{-1}(T) = (T,F,F),(T,F,T),(T,T,F),(F,F,F),(F,F,T) \\
                    & B^{-1}(T) = (T,T,F),(F,T,F),(F,F,F),(T,F,F),(T,F,T),(F,F,T) \\              
                \end{aligned}
            \end{equation*}
            so the conclusion is \textbf{true} .
            \item 
            \begin{equation*}
                \begin{aligned}
                    & \quad (p \lor \lnot q) \land (r \to \lnot p) \Rightarrow (\lnot r \to \lnot q) \\    
                    & (p \land \lnot q) \land ( \lnot r \lor \lnot q) \land \lnot r \land q \\
                    & \equiv (p \land \lnot q) \land ((\lnot r \land \lnot r) \lor (\lnot q \land \lnot r)) \land q \\
                    & \equiv (p \land \lnot q ) \land q \\
                    & \equiv F
                \end{aligned}
            \end{equation*}
            so the conclusion is \textbf{true} .
            \item 
            \begin{equation*}
                \begin{aligned}
                    & \quad (p \lor \lnot q) \land (r \to \lnot p) \Rightarrow (\lnot p \lor\lnot r) \\ 
                    & let \quad A = ( p \lor \lnot q) \land (r \to \lnot p) ,  B = (\lnot p \lor\lnot r) \\
                    & A^{-1}(F) = (T,T,T),(T,F,T),(F,T,F),(F,T,T)\\
                    & B^{-1}(F) = (T,T,T),(T,F,T) \\    
                \end{aligned}
            \end{equation*}
            so the conclusion is \textbf{true} .
            \item 
            \begin{equation*}
                \begin{aligned}
                    & \quad (p \lor \lnot q) \land (r \to \lnot p) \Rightarrow \lnot q \to (\lnot r \lor \lnot p) \\ 
                    & \lnot q \to (\lnot r \lor \lnot p)  \equiv q \lor  \lnot r \lor \lnot p\\
                    & let \quad A = ( p \lor \lnot q) \land (r \to \lnot p) ,  B =  q \lor  \lnot r \lor \lnot p\\
                    & A^{-1}(F) = (T,T,T),(T,F,T),(F,T,F),(F,T,T)\\
                    & B^{-1}(F) = (T,F,T) \\    
                \end{aligned}
            \end{equation*}
            so the conclusion is \textbf{true} .
        \end{enumerate}
    \end{solution*}


    \end{enumerate}
    \end{document}